%% The MIT License (MIT)
%%
%% Copyright (c) 2015 Daniil Belyakov
%%
%% Permission is hereby granted, free of charge, to any person obtaining a copy
%% of this software and associated documentation files (the "Software"), to deal
%% in the Software without restriction, including without limitation the rights
%% to use, copy, modify, merge, publish, distribute, sublicense, and/or sell
%% copies of the Software, and to permit persons to whom the Software is
%% furnished to do so, subject to the following conditions:
%%
%% The above copyright notice and this permission notice shall be included in all
%% copies or substantial portions of the Software.
%%
%% THE SOFTWARE IS PROVIDED "AS IS", WITHOUT WARRANTY OF ANY KIND, EXPRESS OR
%% IMPLIED, INCLUDING BUT NOT LIMITED TO THE WARRANTIES OF MERCHANTABILITY,
%% FITNESS FOR A PARTICULAR PURPOSE AND NONINFRINGEMENT. IN NO EVENT SHALL THE
%% AUTHORS OR COPYRIGHT HOLDERS BE LIABLE FOR ANY CLAIM, DAMAGES OR OTHER
%% LIABILITY, WHETHER IN AN ACTION OF CONTRACT, TORT OR OTHERWISE, ARISING FROM,
%% OUT OF OR IN CONNECTION WITH THE SOFTWARE OR THE USE OR OTHER DEALINGS IN THE
%% SOFTWARE.

% The font could be set to Windows-specific Calibri by using the 'calibri' option
\documentclass[]{mcdowellcv}

% For mathematical symbols
\usepackage{amsmath}
\usepackage{hyperref}


% Set applicant's personal data for header
\name{Eric Jagodinski}
\address{101 N. Beach Rd \linebreak Dania Beach, Florida 33004}
\webpage{\href{www.jagodinski.com}{www.jagodinski.com}}
\contacts{(704) 608-2871 \linebreak \href{ejagodin@fau.edu}{ejagodin@fau.edu}}
\github{}

\begin{document}

	% Print the header
	\makeheader
	
	% Print the content

\begin{cvsection}{Education}
		\begin{cvsubsection}{Florida Atlantic University}{}{2017 -- Dec 2022*}{SeaTech Research Center, Dania Beach, FL}
			\begin{itemize}
				\item \textbf{Candidate for PhD in Ocean Engineering}
				\item \textbf{Dissertation:} ``Data-Driven Identification and Control of Turbulent Structures using Deep Neural Networks'' - Reinforcement Learning-based control in turbulent fluids simulations using Convolutional Neural Networks and Long Short-Term Memory for drag reduction.
			\end{itemize}
		\end{cvsubsection}
		\begin{cvsubsection}{Florida Atlantic University}{}{2018}{SeaTech Research Center, Dania Beach, FL}
			\begin{itemize}
				\item \textbf{M.S. in Ocean Engineering} Earned En Passant while completing courses for my PhD
			\end{itemize}
		\end{cvsubsection}				
		\begin{cvsubsection}{Florida Atlantic University}{}{2010 -- 2016}{Boca Raton, FL}
			\begin{itemize}
				\item \textbf{Bachelor of Science in Ocean Engineering}
				\item \textbf{Capstone Project:} Full system designed and built prototype of an autonomous surface vehicle capable of GPS navigation and station keeping in dynamic conditions (Electrical Team Lead).
			\end{itemize}
		\end{cvsubsection}
	\end{cvsection}
	
			\begin{cvsection}{Skills}
		\begin{cvsubsection}{}{}{}	
			\begin{itemize}
				\item Fortran, Python, C++, MATLAB, R, SQL
				\item TensorFlow, Keras, Artificial Neural Networks (CNN, LSTM), Deep Reinforcement Learning, Principal Component Analysis
				\item Git, Linux, Bash, HPC, MPI, ARM DDT (parallel MPI debugger), Slurm, Tableau
			\end{itemize}
		\end{cvsubsection}
	\end{cvsection}
	

	\begin{cvsection}{Professional Experience}
		\begin{cvsubsection}{Course Instructor}{}{2021}{Florida Atlantic University, Boca Raton, FL}
			%iChat AV			
			\begin{itemize} 
			\item Taught an undergraduate Fluid Mechanics course (EML3701) to a class of 30 students (in-person and remote).
			\end{itemize}
		\end{cvsubsection}
		
		\begin{cvsubsection}{Graduate Intern}{}{2018}{Naval Research Laboratory, Stennis Space Center, MS}
			\begin{itemize}
            	\item Developed simulations using an open source CFD software for rogue wave and wind interaction using High-Performance Computing.
			\end{itemize}
		\end{cvsubsection}
		
			\begin{cvsubsection}{Engineering Technician}{}{2014-2015}{Agilis Engineering, Palm Beach Gardens, FL}
			\begin{itemize}
            	\item Assembled computer monitoring and signal conditioning systems used on GE turbines for NextEra and analyzed real-time turbine data for monthly reports for Florida Power and Light.
			\end{itemize}
		\end{cvsubsection}
	\end{cvsection}
	
		\begin{cvsection}{Research}
		\begin{cvsubsection}{Publications}{}{}
			\begin{itemize}
				\item Jagodinski, E., Zhu, X., Verma, S., \textbf{Data-driven identification of dynamically important regions in turbulent flows using 3D Convolutional Neural Networks} (\textit{submitted})  Autonomously identified critical regions in turbulent flow using 3D convolutional neural networks and a custom modified interpretation technique. Applied advanced data science methods to analyze efficacy of the technique.
			\end{itemize}
		\end{cvsubsection}
		
		\begin{cvsubsection}{Conference Presentations}{}{}
			\begin{itemize}
				\item  \textbf{Data-Driven blowing-suction control in a turbulent channel flow.}  APS Division of Fluid Dynamics (2021)
				\item \textbf{Convolutional neural networks for identifying coherent turbulent structures.} APS Division of Fluid Dynamics (2019)
			\end{itemize}
		\end{cvsubsection}
	\end{cvsection}
	
		\begin{cvsubsection}{Conference Posters}{}{}
			\begin{itemize}
				\item  \textbf{Turbulent flow identification using 3D convolutional neural networks} FAU Data-Driven Science and AI Conference (2022)
			\end{itemize}
		\end{cvsubsection}
	
	\begin{cvsection}{Certifications}
		\begin{cvsubsection}{}{}{}	
			\begin{itemize}
				\item \textbf{Google Data Analytics Specialization} (2022) A professional certificate through Coursera on preparing, processing, analyzing and presenting data. (SQL, Tableau, R)
				\item \textbf{Offshore Engineering Graduate Certificate} (2018) A graduate level specialization. Courses: Advanced Hydrodynamics, Offshore Structures, Hydrodynamics of Ship Design.
			\end{itemize}
		\end{cvsubsection}
	\end{cvsection}
	
	
\end{document}